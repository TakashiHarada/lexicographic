%定義

%\input{setting}

%\documentclass[11pt,twocolumn]{jarticle} 
\documentclass[11pt,dvipdfmx]{jarticle} 
\usepackage{graphicx}
\usepackage{fancybox}
\usepackage{comment}
\usepackage{amsmath}
\usepackage{amssymb}
\usepackage{amsfonts}
\usepackage{amsthm}
\usepackage{euler}
\usepackage{color}
%\usepackage{theorem}
%\usepackage{proof}
\usepackage[deluxe]{otf}
%\usepackage{mathptmx}

%\newtheorem{def}{定義}[section]

\pagestyle{empty}

%余白とか

\setlength{\topmargin}{-2cm} 
\setlength{\textheight}{26.5cm} 
\setlength{\textwidth}{18.5cm}
\setlength{\oddsidemargin}{-1.3cm} 
\setlength{\columnsep}{.5cm}
\newcommand{\noin}{\noindent}
\catcode`@=\active \def@{\hspace{0.9bp}-\hspace{0.9bp}}
\theoremstyle{definition}
\newtheorem{definition}{定義}[section]
\newtheorem{proposition}{命題}[section]
\newtheorem{theorem}{定理}[section]
\newtheorem{lemma}{補題}[section]

%タイトル
\title{辞書式順序が整列順序であることの証明}
\setcounter{footnote}{1}
\author{原田 崇司\if0\thanks{神奈川大学大学院理学研究科情報科学先攻}\fi}
\date{\today}
\西暦

%タイトル作成

\begin{document}

\maketitle
\thispagestyle{empty}

\section{整列順序}
%\begin{def}
% first definition
%\end{def}
集合$S$上の二項関係$\prec$が以下の四つの性質を満たす時,二項関係$\prec$は集合$S$上の整列順序関係である.
\begin{enumerate}
 \item $\forall x. \ x \in S \Rightarrow x \prec x$
 \item $\forall x. \, \forall y. \ x \in S \Rightarrow y \in S \Rightarrow x \prec y \Rightarrow y \prec x \Rightarrow x = y$
 \item $\forall x. \, \forall y. \, \forall z. \ x \in S \Rightarrow y \in S \Rightarrow z \in S \Rightarrow x \prec y \Rightarrow y \prec z \Rightarrow x \prec z$
 \item $\forall A. \, (A \neq \emptyset \land A \subset S) \Rightarrow \exists u. \, u \in A \land \forall x \, (x \in A \Rightarrow u \prec x)$
\end{enumerate}

\section{辞書式順序}
The Art of Computer Programming Volume1 の p.20 では,辞書式順序を下記のように定義している.

\begin{definition}
Let $S$ be well-ordered by $\prec$ and for $n > 0$ let $T_{n}$ be the set of all $n$--tuples $(x_{1}, x_{2}, \dots , x_{n})$ of elements $x_{j}$ in $S$. Define $(x_{1}, x_{2}, \dots, x_{n}) \prec (y_{1}, y_{2}, \dots, y_{n})$, if there is some $k$, $1 \leq k \leq n$, such that $x_{j} = y_{j}$ for $1 \leq j < k$, but $x_{k} \prec y_{k}$ in $S$. 
 \label{dfn:lex}
\end{definition}
%%\begin{align*}
上記の辞書式順序$\lhd_{n}$の定義を論理式で表すと下記のようになる.
\begin{equation}
%X \lhd_{n} Y \overset{\mathrm{def}}{=} \forall n. \, n \in \mathbb{N} \Rightarrow \exists k. \, k \in \mathbb{N} \land (1 \leq k \leq n) \land \Bigl( \forall j. \ j \in \mathbb{N} \Rightarrow (1 \leq j < k) \Rightarrow  x_{j} = y_{j} \Bigr) \land \Bigl( x_{k} \prec y_{k} \Bigr)
X \lhd_{n} Y \overset{\mathrm{def}}{=} \exists k. \, k \in \mathbb{N} \land (1 \leq k \leq n) \land \Bigl( \forall j. \ j \in \mathbb{N} \Rightarrow (1 \leq j < k) \Rightarrow  x_{j} = y_{j} \Bigr) \land \Bigl( x_{k} \prec y_{k} \Bigr)
\end{equation}
%% & \forall n \in \mathbb{N} \ \exists k \Biggl( \biggl(1 \leq k \ \land \ k \leq n \biggr) \ \land \ \biggl( \forall j \ \Bigl( (1 \leq j \ \land \ j < k ) \Rightarrow x_{j} = y_{j} \Bigr) \ \land \ (x_{k} \prec y_{k}) \biggr) \Biggr)
%%\end{align*}
ただし,$X \equiv (x_{1}, x_{2}, \dots, x_{n}), Y \equiv (y_{1}, y_{2}, \dots, y_{n})$.

\section{辞書式順序$\lhd_{n}$が整列順序であることの証明}
辞書式順序$\lhd_{n}$が整列順序の性質4つを満たすことを命題3.1,命題3.2,命題3.3,命題3.4を示すことで示す.

\begin{proposition}
\[
\forall X. \ X \in T_{n} \Rightarrow X \lhd_{n} X
\]
%ここで,$\mathscr{S}$は定義\ref{dfn:lex}にある$T_{n}$とする.
\end{proposition}

\begin{proof}
\begin{equation}
% \forall X. \ X \in \mathscr{S} \Rightarrow X \lhd_{n} X
 \forall X. \ X \in T_{n} \Rightarrow X \lhd_{n} X
 \label{eq:one}
\end{equation}
を示す.\par
%\noindent $\mathscr{S}$の任意の要素$X^{'}$をとる.\par
\noindent $T_{n}$の任意の要素$X^{'}$をとる.\par
\begin{equation}
 X^{'} \lhd_{n} X^{'}
 \label{eq:one1}
\end{equation}
を示す.\par
\noindent $\lhd_{n}$の定義より
\begin{equation}
\exists k. \, k \in \mathbb{N} \land (1 \leq k \leq n) \land \Bigl( \forall j. \ j \in \mathbb{N} \Rightarrow (1 \leq j < k) \Rightarrow  x_{j} = x_{j} \Bigr) \land \Bigl( x_{k} \prec x_{k} \Bigr)
\end{equation}
を示せばよい.\par
\noindent $k$として$n$をとる.
\begin{equation}
\Bigl( \forall j. \ j \in \mathbb{N} \Rightarrow (1 \leq j < n) \Rightarrow  x_{j} = x_{j} \Bigr) \land \Bigl( x_{n} \prec x_{n} \Bigr)
 \label{eq:one2}
\end{equation}
を示す.\par
\noindent (\ref{eq:one2})の$\land$の左側の命題は$S$上の$=$の反射律より正しく,$\land$の右側の命題は$S$上の$\prec$の反射律より正しいので,(\ref{eq:one1})は正しい.ここで,(\ref{eq:one1})の$X^{'}$は任意だったので(\ref{eq:one})が示された.
\end{proof}

\begin{proposition}
\[
\forall X. \ \forall Y. \ \ X \in T_{n} \Rightarrow Y \in T_{n} \Rightarrow X \lhd_{n} Y \Rightarrow Y \lhd_{n} X \Rightarrow X = Y
\]
%ここで,$T_{n}$は定義\ref{dfn:lex}にある$T_{n}$の全体とする.
\color{red}{ここで,$X = Y \overset{\mathrm{def}}{=} \forall i. \ i \in \mathbb{N} \Rightarrow 1 \leq i \leq n \Rightarrow x_{i} = y_{i}$}\footnote{$X = Y$の定義はThe Art of Computer Programming自体には載っていないので,それらしいものを勝手に定義した.}
\end{proposition}

\begin{proof}
\begin{equation}
\forall X. \ \forall Y. \ \ X \in T_{n} \Rightarrow Y \in T_{n} \Rightarrow X \lhd_{n} Y \Rightarrow Y \lhd_{n} X \Rightarrow X = Y
 \label{eq:two}
\end{equation}
を示す.\par
\noindent $T_{n}$の任意の要素$X^{'},Y^{'}$をとる.
\begin{equation}
X^{'} \lhd_{n} Y^{'} \Rightarrow Y^{'} \lhd_{n} X^{'} \Rightarrow X^{'} = Y^{'}
 \label{eq:two'}
\end{equation}
を示す.\par
\noindent (\ref{eq:two'})を示すには$X^{'} \lhd_{n} Y^{'}$と$Y^{'} \lhd_{n} X^{'}$を仮定して$X^{'} = Y^{'}$を示せば充分である.
\begin{equation}
 X^{'} \lhd_{n} Y^{'} \hspace{5mm} \Biggl( \exists k. \, k \in \mathbb{N} \land (1 \leq k \leq n) \land \Bigl( \forall j. \ j \in \mathbb{N} \Rightarrow (1 \leq j < k) \Rightarrow  x_{j} = y_{j} \Bigr) \land \Bigl( x_{k} \prec y_{k} \Bigr) \Biggr)
 \label{eq:xleqy}
\end{equation}
\begin{equation}
 Y^{'} \lhd_{n} X^{'} \hspace{5mm} \Biggl( \exists l. \, l \in \mathbb{N} \land (1 \leq l \leq n) \land \Bigl( \forall j. \ j \in \mathbb{N} \Rightarrow (1 \leq j < l) \Rightarrow  y_{j} = x_{j} \Bigr) \land \Bigl( y_{l} \prec x_{l} \Bigr) \Biggr)
 \label{eq:yleqx} 
\end{equation}
を仮定する.
\begin{equation}
 X^{'} = Y^{'}
\end{equation}
を示す. \par
\noindent 仮定(\ref{eq:xleqy}),(\ref{eq:yleqx})より,$k = l = n$?なので
\begin{equation}
\Bigl( \forall j. \ j \in \mathbb{N} \Rightarrow (1 \leq j < n) \Rightarrow  y_{j} = x_{j} \Bigr) \land \Bigl( x_{n} \prec y_{n} \ \land \ y_{n} \prec x_{n} \Bigr)
\end{equation}
が成り立つ.\par
\noindent !!!!!!!!!!!!!!!!!!!!!!!!!!!!!!!!!!!!!!!!!!!!!! まだ$k = l = n$の所を証明できていません.!!!!!!!!!!!!!!!!!!!!!!!!!!!!!!!!!!!!!!!!!!!!!! 
\begin{align*}
 & \Bigl( \forall j. \ j \in \mathbb{N} \Rightarrow (1 \leq j < n) \Rightarrow  y_{j} = x_{j} \Bigr) \land \Bigl( x_{n} \prec y_{n} \ \land \ y_{n} \prec x_{n} \Bigr) \\
\iff & \Bigl( \forall j. \ j \in \mathbb{N} \Rightarrow (1 \leq j < n) \Rightarrow  y_{j} = x_{j} \Bigr) \land \Bigl( x_{n} = y_{n} \Bigr) \\
\iff & \ \ \forall j. \ j \in \mathbb{N} \Rightarrow (1 \leq j \leq n) \Rightarrow  y_{j} = x_{j} \\
\iff & \ \ X = Y
\end{align*}
よって,(\ref{eq:two'})が成り立つ.(\ref{eq:two'})で$X^{'}$と$Y^{'}$は任意であったので(\ref{eq:two})が示された.
\end{proof}

\end{document}
